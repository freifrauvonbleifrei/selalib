\documentclass[11pt,a4paper]{amsart}
\usepackage[dvips]{epsfig}
\usepackage{graphics}
\usepackage{latexsym}
\usepackage{verbatim}
\usepackage{amsmath}
\usepackage{amsthm}
\usepackage{amssymb}
%\usepackage{MnSymbol}
%\usepackage{stmaryrd}
\usepackage{hyperref}
\usepackage[]{hyperref}
\hypersetup{
    colorlinks=true,%
    filecolor=black,%
    linkcolor=black,%
    urlcolor=black  %
}
%%%%%%%%%%% plus plus %%%%%%%%%%%%%
\usepackage [table]{xcolor}
%\usepackage{multirow}
%\usepackage{float}
%\usepackage{tikz}
%\usepackage{subfig}
%\usepackage{algorithm}
%\usepackage{algpseudocode}
%\usepackage{algorithm2e}
%%%%%%%%%%%%%%%%%%%%%%%%%%%%%%%%%%%%%%%%%%%%%%%%%%%%%%%%%%%%%%%%%%%%%%%%%%%
%%% Global Settings %%%%%%%%%%%%%%%%%%%%%%%%%%%%%%%%%%%%%%%%%%%%%%%%%%%%%%%
%%%%%%%%%%%%%%%%%%%%%%%%%%%%%%%%%%%%%%%%%%%%%%%%%%%%%%%%%%%%%%%%%%%%%%%%%%%

\graphicspath{{images/}}	% Root directory of the pictures 
\tracingstats=2		% Enabled LaTeX logging with conditionals

\setcounter{tocdepth}{2}


%%%%%%%%%%%%%%%%%%%%%%%%%%%%%%%%%%%%%%%%%%%%%%%%%%%%%%%%%%%%%%%%%%%%%%%%%%%
%%% Environments    %%%%%%%%%%%%%%%%%%%%%%%%%%%%%%%%%%%%%%%%%%%%%%%%%%%%%%%
%%%%%%%%%%%%%%%%%%%%%%%%%%%%%%%%%%%%%%%%%%%%%%%%%%%%%%%%%%%%%%%%%%%%%%%%%%%

\newtheorem{theorem}{Theorem}[section]
\newtheorem{lemma}[theorem]{Lemma}
\newtheorem{corollary}[theorem]{Corollary}
\newtheorem{proposition}[theorem]{Proposition}
\newtheorem{definition}[theorem]{Definition}
\newtheorem{remark}[theorem]{Remark}

\def\theproposition {{\arabic{section}.\arabic{theorem}}}
\def\thetheorem {{\arabic{section}.\arabic{theorem}}}
\def\thelemma {{\arabic{section}.\arabic{theorem}}}
\def\thecorollary {{\arabic{section}.\arabic{theorem}}}
\def\thedefinition {{\arabic{section}.\arabic{theorem}}}
\def\theremark {{\arabic{section}.\arabic{theorem}}}
\def\theequation {\arabic{section}.\arabic{equation}}
\def\Examples{\medskip\noindent{\bf Examples: }}
\def\Remark{\medskip\noindent{\bf Remark: }}
\def\Remarks{\medskip\noindent{\bf Remarks: }}

\def\bul{{$\bullet$\hspace*{2mm}}}


\newcommand\CC{\hbox{C\kern -.58em {\raise .54ex \hbox{$\scriptscriptstyle |$}}
  \kern-.55em {\raise .53ex \hbox{$\scriptscriptstyle |$}} }}
%\newcommand\qed{\hfill$\sqcap\kern-8.0pt\hbox{$\sqcup$}$}
\newcommand\NN{\hbox{I\kern-.2em\hbox{N}}}
\newcommand\RR{\mathbb{R}}
\newcommand\ZZ{{{\rm Z}\kern-.28em{\rm Z}}}
\newcommand\Gradx{ \nabla_{\mathbf{x}}}
\newcommand\Gradv{ \nabla_{\mathbf{v}}}
\newcommand\Div{ \textrm{div}}
\newcommand\Rot{ \textrm{curl}}
\newcommand\jj{\mathbf{J}}
\newcommand\xx{ \mathbf{x} }
\newcommand\vv{ \mathbf{v} }

%\usepackage{algorithmic}
\newcommand{\ens}[1]{\mathbb{#1}}

%%%%%%%%%%%%%%%%%%%%%%%%%%%%%%%%%%%%%%%%%%%%%%%%%%%%%%%%%%%%%%%%%%%%%%%%%%%
\setlength{\oddsidemargin}{.25cm} \setlength{\evensidemargin}{.25cm}
\setlength{\textwidth}{15.5cm} \setlength{\textheight}{23.5cm}
\setlength{\topmargin}{0.25cm}


%%%%%%%%%%%%%%%%%%%%%%%%%%%%%%%%%%%%%%%%%%%%%%%%%%%%%%%%%%%%%%%%%%%%%%%%%%%
%%% Definitions   %%%%%%%%%%%%%%%%%%%%%%%%%%%%%%%%%%%%%%%%%%%%%%%%%%%%%%%%%
%%%%%%%%%%%%%%%%%%%%%%%%%%%%%%%%%%%%%%%%%%%%%%%%%%%%%%%%%%%%%%%%%%%%%%%%%%%


\def\signcy{\bigskip\bigskip\hspace{80mm}
\vbox{{\sc Charles prouveur \par\vspace{3mm}
Universit\'e de Lyon,\par
UMR5208, Institut Camille Jordan,\par
43 boulevard 11 novembre 1918,\par
F-69622 Villeurbanne cedex,  FRANCE
\par\vspace{3mm}e-mail:} prouveur@math.univ-lyon1.fr }}

%*****************************************
% GLOBALS : packages, variable definitions, 
%           spacings and more.
%*****************************************
%%% TeX-master: "thesis_main.tex"

\usepackage[english]{babel}
\selectlanguage{english}
\usepackage{amsmath,epsfig,graphics,graphicx}
\usepackage{amsfonts,amssymb,amsbsy}
\usepackage{xcolor,color}
\usepackage{enumitem}
\usepackage{array}
\usepackage{float}  % replaced the package 'here' (rk:  {figure}[H] normally equivalent to {figure}[ht])
\usepackage{verbatim}
\usepackage{hyperref,eso-pic}
\usepackage{caption}
\usepackage{subcaption}


%---> page properties
\newdimen\decalage
\paperheight=29.7 true cm \paperwidth=21 true cm
\textheight=24.2 true cm \textwidth=15 true cm
\decalage=0.25 true cm

% automatic computation of \topmargin,
% \evensidemargin and \oddsidemargin
\oddsidemargin=\paperwidth
\advance\oddsidemargin by -\textwidth
\divide\oddsidemargin by 2
\advance\oddsidemargin by -1 in
\evensidemargin=\oddsidemargin
\advance\oddsidemargin by \decalage 
\advance\evensidemargin by -\decalage

\topmargin=\paperheight
\advance\topmargin by -\headheight
\advance\topmargin by -\headsep
\advance\topmargin by -\textheight
\advance\topmargin by -\footskip
\divide\topmargin by 2
\advance\topmargin by -1 in


% ---> renew commands of headers (section, paragraph, ...)
%\RequirePackage[calcwidth]{titlesec}
%
%\titleformat{\section}[hang]{\bfseries}
%{\large\thesection}{12pt}{\Large}[{\titlerule[0.5pt]}]
%
%\RequirePackage[calcwidth]{titlesec}
%
%\titleformat{\paragraph}[hang]{\sffamily\bfseries}
%{\Huge\theparagraph}{12pt}{\Huge}[{\titlerule[0.5pt]}]
%
%\titleformat{\subparagraph}[hang]{\bfseries}
%{\tiny\thesubparagraph}{12pt}{\Large\flushright}


%Figures path : 
\newcommand{\figurespath}{./figures}
%\numberwithin{equation}{section}

%---> general definitions
\newcommand{\gysela}{\textsc{Gysela }}
\def\NN{\mathbb{N}}
\def\RR{\mathbb{R}}
\def\ZZ{\mathbb{Z}}
\def\ZZd{\mathbb{Z}^2}



\newcommand{\eg}{{\it e.g. }}
\newcommand{\etal}{{\em et al }}
% --> LM's additions :
\newcommand{\ie}{{\it i.e.  }}


%---> personal environment
\newenvironment{tocheck}%
{\color{blue}\noindent[TO CHECK]}

\newenvironment{todo}%
{\color{magenta}\noindent[TODO]}

\newenvironment{becareful}%
{\color{red}\noindent[ATTENTION]}

\newenvironment{question}%
{\color{red}\noindent[QUESTION]}

% --> LM's addition : for code environment
\definecolor{Zgris}{rgb}{0.87,0.85,0.85}
\newsavebox{\BBbox}
\newenvironment{DDbox}[1]{
\begin{lrbox}{\BBbox}\begin{minipage}{\linewidth}}
{\end{minipage}\end{lrbox}\noindent\colorbox{Zgris}{\usebox{\BBbox}} \\
[.5cm]}



%---> personal command
\newcommand{\ask}[1]{\textcolor{red}{#1}}
\newcommand{\rhs}{right hand side }
\newcommand{\rmk}[1]{{\textcolor{red}{\underline{Remark #1:} }}}

%---> for code description
\newcommand{\code}[1]{\textcolor{darkgray}{\texttt{#1}}}
\newcommand{\codecomment}[1]{\textcolor{darkgray}{\underline{Code comment:} #1}}

%----> for information about the programmation
\newcommand{\numdiag}{$<$\texttt{num\_diag}$>$}

%---> mathematic operators
\newcommand{\vecnabla}{{\pmb\nabla}}
\newcommand{\grad}{\vecnabla}
\newcommand{\rot}{\vecnabla\times}
\newcommand{\rotational}{\overrightarrow{rot} \:}
\newcommand{\diverg}{\grad\cdot}
\newcommand{\divergence}{{\rm div}}
\newcommand{\ee}{{\rm e}}
\newcommand{\norm}[1]{\|{#1}\|}
% --> LM's additions : in patch/logical domain :
\newcommand{\gradphy}{\nabla_{\Phy} \:}
\newcommand{\rotphy}{\nabla_{\Phy}\times}
\newcommand{\rotationalphy}{\overrightarrow{rot_{\Phy}}\:}
\newcommand{\divergphy}{\nabla_{\Phy}\cdot}
% --> LM's additions : in physical domain
\newcommand{\gradpat}{\nabla_{\Pat} \:}
\newcommand{\rotpat}{\nabla_{\Pat}\times}
\newcommand{\rotationalpat}{\overrightarrow{rot_{\Pat}}\:}
\newcommand{\divergpat}{\nabla_{\Pat}\cdot}
% --> LM's additions : other common operators
\newcommand{\gradvit}{\nabla_{\mathbf{v}} \:}



%--> Poisson Brackets
%\newcommand{\PoissBrack}[1]{\left[#1\right]}
%\newcommand{\PoissBrackgyro}[1]{\left\{#1\right\}}

%--> GYSELA equations
%\newcommand{\Bparstar}{B_{\|s}^*}
%\newcommand{\Bparstarindx}[1]{B_{\|s,{#1}}^*}
%\newcommand{\Kshear}{K_{\rm shear}}
%\newcommand{\KgradB}{K_{\nabla B}}
%\newcommand{\gradpar}{\grad_{\|}^{\ast}}
%\newcommand{\phibar}{\bar{\phi}}
%\newcommand{\pot}{U}
%\newcommand{\potbar}{\bar{U}}
%\newcommand{\rotb}{\rot b}
%\newcommand{\vGpar}{v_{G\|}}
%\newcommand{\vGparj}{v_{G\|,j}}
\newcommand{\vecA}{\mathbf{A}}
\newcommand{\vecB}{\mathbf{B}}
\newcommand{\vecBstar}{\vecB^{\ast}_s}
\newcommand{\vecE}{\mathbf{E}}
\newcommand{\vecGamma}{\mathbf{\Gamma}}
\newcommand{\vecN}{\mathbf{N}}
\newcommand{\vecJ}{\mathbf{J}}
\newcommand{\vecR}{\mathbf{R}}
\newcommand{\vecU}{\mathbf{U}}
\newcommand{\vecV}{\mathbf{V}}
\newcommand{\vecW}{\mathbf{W}}
\newcommand{\vecX}{\mathbf{X}}
\newcommand{\vecb}{\mathbf{b}}
\newcommand{\vecbstar}{\mathbf{b}_s^{\ast}}
\newcommand{\vecd}{\mathbf{d}}
\newcommand{\vece}{\mathbf{e}}
\newcommand{\vecj}{\mathbf{j}}
\newcommand{\veck}{\mathbf{k}}
\newcommand{\vecm}{\mathbf{m}}
\newcommand{\vecp}{\mathbf{p}}
\newcommand{\vecr}{\mathbf{r}}
\newcommand{\vecrho}{\pmb{\rho}}
\newcommand{\vecrhos}{\pmb{\rho}_s}
\newcommand{\vecu}{\mathbf{u}}
\newcommand{\vecv}{\mathbf{v}}
\newcommand{\vecx}{\mathbf{x}}
\newcommand{\vecxG}{\mathbf{x}_G}
\newcommand{\vperp}{v_\perp}
\newcommand{\xGi}{x_G^i}

%--> LM's variables addition :
\newcommand{\advec}{\vecA}
\newcommand{\patadvec}{\tilde{\advec}}
\newcommand{\advecvar}[1]{a_{#1}}
\newcommand{\patadvecvar}[1]{\tilde{a}_{#1}}
\newcommand{\Phy}{\mathbf{x}}
\newcommand{\Pat}{\boldsymbol{\eta}}
\newcommand{\dist}{f}
\newcommand{\patdist}{\tilde{\dist}}



%--> parameters
%\newcommand{\rhostar}{\rho_*}
%\newcommand{\nustar}{\nu_*}
%\newcommand{\nustars}{\nu_{*s}}
%\newcommand{\nustarsprime}{\nu_{*s'}}
%\newcommand{\kperp}{k_\perp}

%---> for gyro-averaged quantities and gyroaverage operator
%\newcommand{\gbar}{\bar{g}}
%\newcommand{\gbarm}[1]{\bar{g}^{m}_{#1}}
%\newcommand{\gm}[1]{g^{m}_{#1}}
%\newcommand{\gyrophi}{\phibar}
%\newcommand{\Fbars}{\bar{F}_s}
%\newcommand{\gyroFbars}{J_0\bar{F}_s}
%\newcommand{\Fbarsprime}{\bar{F}_{s'}}
%\newcommand{\Fbarseq}{\bar{F}_{s,{\rm eq}}}
%\newcommand{\Jphi}{\mathcal{J}\cdot\phi}
 
%---> for electrons
%\newcommand{\nezero}{n_{e_0}}

%---> for ion species
%\newcommand{\cs}{c_s}
%\newcommand{\ms}{m_s}
%\newcommand{\ns}{n_s}
%\newcommand{\nseq}{n_{s,{\rm eq}}}
%\newcommand{\nszero}{n_{s_0}}
%\newcommand{\nGs}{n_{G_s}}
%\newcommand{\nGseq}{n_{G_s,{\rm eq}}}
%\newcommand{\nspol}{n_{s,{\rm pol}}}
%\newcommand{\qs}{q_s}
%\newcommand{\Ds}{D_s}
%\newcommand{\Fs}{F_s}
%\newcommand{\FMs}{F_{M_s}}
%\newcommand{\Fseq}{F_{s,{\rm eq}}}
%\newcommand{\Ts}{T_s}
%\newcommand{\Zs}{Z_s}
%\newcommand{\Zsprime}{Z_{s^\prime}}
%\newcommand{\rhoLs}{\rho_{L_s}}
%\newcommand{\vecrhoLs}{\vecrho_{L_s}}
%\newcommand{\Omegas}{\Omega_s}

%---> invariance and normalization
%\newcommand{\alphacoll}{\alpha_{\rm coll}}
%\newcommand{\alphadiff}{\alpha_{\rm diff}}
%\newcommand{\alphakrook}{\alpha_{\rm Krook}}
%\newcommand{\hatb}{\hat{b}}
%\newcommand{\hatepsilon}{\hat{\epsilon}}
%\newcommand{\hatchi}{\hat{\chi}}
%\newcommand{\hatddt}[1]{\frac{{\rm d}#1}{{\rm d}\hat{t}}}
%\newcommand{\hatnabla}{\hat{\nabla}}
%\newcommand{\hatgrad}{\hat{\grad}}
%\newcommand{\hatgradF}{\hat{\grad} F}
%\newcommand{\hatgradG}{\hat{\grad} G}
%\newcommand{\hatgradpar}{\hatgrad_{\|}^{\ast}}
%\newcommand{\hatjacobs}{\hat{\cal J}_{\rm x}}
%\newcommand{\hatjacobv}{\hat{\cal J}_{\rm v}}
%\newcommand{\hatl}{\hat{l}}
%\newcommand{\hatms}{\hat{m}_s}
%\newcommand{\hatns}{\hat{n}_s}
%\newcommand{\hatnezero}{\hat{n}_{e_0}}
%\newcommand{\hatnszero}{\hat{n}_{s_0}}
%\newcommand{\hatnseq}{\hat{n}_{s,{\rm eq}}}
%\newcommand{\hatnGs}{\hat{n}_{G_s}}
%\newcommand{\hatnGseq}{\hat{n}_{G_s,{\rm eq}}}
%\newcommand{\hatnu}{\hat{\nu}}
%\newcommand{\hatpartial}{\hat{\partial}}
%\newcommand{\hatphi}{\hat{\phi}}
%\newcommand{\hatphibar}{\hat{\phibar}}
%\newcommand{\hatn}{\hat{n}}
%\newcommand{\hatmu}{\hat{\mu}}
%\newcommand{\hatq}{\hat{q}}
%\newcommand{\hatr}{\hat{r}}
%\newcommand{\hatrho}{\hat{\rho}}
%\newcommand{\hatrmin}{\hatr_{\rm min}}
%\newcommand{\hatrmax}{\hatr_{\rm max}}
%\newcommand{\hatt}{\hat{t}}
%\newcommand{\hattheta}{\hat{\theta}}
%\newcommand{\hatthreeD}{\hatr,\theta,\varphi}
%\newcommand{\hatv}{\hat{v}}
%\newcommand{\hatvecb}{\hat{\vecb}}
%\newcommand{\hatvecxG}{\hat{\vecx}_G}
%\newcommand{\hatvecbstar}{\hat{\mathbf{b}}_s^{\ast}}
%\newcommand{\hatvecv}{\hat{\vecv}}
%\newcommand{\hatvs}{\hat{v}_s}
%\newcommand{\hatvTs}{\hat{v}_{T_s}}
%\newcommand{\hatvTsprime}{\hat{v}_{T_s'}}
%\newcommand{\hatvecB}{\hat{\vecB}}
%\newcommand{\hatvecE}{\hat{\vecE}}
%\newcommand{\hatvecJ}{\hat{\vecJ}}
%\newcommand{\hatvGpar}{\hat{v}_{G\|}}
%\newcommand{\hatA}{\hat{A}}
%\newcommand{\hatAs}{\hat{A}_s}
%\newcommand{\hatAsnew}{\hat{A}_{s ({\rm new})}}
%\newcommand{\hatB}{\hat{B}}
%\newcommand{\hatBparstar}{\hat{B}_{\|s}^*}
%\newcommand{\hatE}{\hat{E}}
%\newcommand{\hatEpars}{\hat{E}_{\parallel_s}}
%\newcommand{\hatEs}{\hat{E}_s}
%\newcommand{\hatEsprime}{\hat{E}_{s^\prime}}
%\newcommand{\hatFs}{\hat{F}_s}
%\newcommand{\hatFseq}{\hat{F}_{s,{\rm eq}}}
%\newcommand{\hatFbars}{\hat{\bar{F}}_s}
%\newcommand{\hatFbarseq}{\hat{\bar{F}}_{s,{\rm eq}}}
%\newcommand{\hatI}{\hat{I}}
%\newcommand{\hatJ}{\hat{J}}
%\newcommand{\hatP}{\hat{P}}
%\newcommand{\hatPhi}{\hat{\Phi}}
%\newcommand{\hatR}{\hat{R}}
%\newcommand{\hatS}{\hat{S}}
%\newcommand{\hatT}{\hat{T}}
%\newcommand{\hatTe}{\hat{T}_e}
%\newcommand{\hatTs}{\hat{T}_s}
%\newcommand{\hatTsprime}{\hat{T}_{s'}}
%\newcommand{\hatV}{\hat{V}}
%\newcommand{\hatX}{\hat{X}}
%\newcommand{\hatZe}{\hat{Z}_e}
%\newcommand{\hatZs}{\hat{Z}_s}
%\newcommand{\hatZsprime}{\hat{Z}_{s'}}
%\newcommand{\hatZsnew}{\hat{Z}_{s ({\rm new})}}
%\newcommand{\vTs}{v_{T_s}}
%\newcommand{\vTsprime}{v_{T_s'}}
%\newcommand{\vnorm}{v_{T_{s0}}}
%\newcommand{\vnormprime}{v_{T_{s^\prime 0}}}

%----> for the dimensions of the problem
%\newcommand{\threeD}{r,\theta,\varphi}
%\newcommand{\fourD}{r,\theta,\varphi,\vGpar}
%\newcommand{\fiveD}{r,\theta,\varphi,\vGpar,\mu}
% --> LM's addition : physical mapping
\newcommand{\phyvar}[1]{x_{#1}}
\newcommand{\phy}{\phyvar{1} , \phyvar{2}}
% --> LM's addition : patch/logical mapping
\newcommand{\patvar}[1]{\eta_{#1}}
\newcommand{\pat}{\patvar{1} , \patvar{2}}
% --> LM's addition : time variables
\newcommand{\tphy}{t, \phy}
\newcommand{\tpat}{t, \pat}
\newcommand{\tvar}[1]{t_{#1}}
% --> LM's addition : direct mapping
\newcommand{\mapp}{F}
\newcommand{\mapping}{\mapp(\pat)}
\newcommand{\mappvar}[1]{\mapp_{#1}}
\newcommand{\mappingvar}[1]{\mapp_{#1}(\pat)}
% --> LM's addition : inverse mapping
\newcommand{\invmapp}{F^{-1}}
\newcommand{\invmapping}{F^{-1}(\phy)}
\newcommand{\invmappvar}[1]{F^{-1}_{#1}}
\newcommand{\invmappingvar}[1]{F^{-1}_{#1}(\phy)}


% --> LM's addition : boundary condition
\newcommand{\patches}[1]{\mathcal{P}_{#1}}
\newcommand{\domain}{\Omega}
\newcommand{\bound}{\partial \Omega}
\newcommand{\boundpatpat}[2]{\partial \Gamma_{#1,#2}}
\newcommand{\boundpat}[1]{\partial \patches{#1}}
\newcommand{\npat}{N_{pat} - 1}


% --> LM's addition : order/complexity
\newcommand{\oden}[1]{$O(n^{#1})$} 


%---> jacobian
%\newcommand{\jacobs}{{\cal J}_{\rm x}}
%\newcommand{\jacobv}{{\cal J}_{\rm v}}
%\newcommand{\jacobspace}{{\cal J}_{\rm x}(r,\theta)}
%\newcommand{\jacobvel}{{\cal J}_{\rm v}(r,\theta,\vGpar)}
% --> LM's addition
\newcommand{\jacob}{\sqrt{g}}
\newcommand{\jacuu}{\derivate{\mappvar{1}}{1}}
\newcommand{\jacud}{\derivate{\mappvar{2}}{1}}
\newcommand{\jacdu}{\derivate{\mappvar{1}}{2}}
\newcommand{\jacdd}{\derivate{\mappvar{2}}{2}}

%-----> for derivatives
\newcommand{\ddr}[1]{\frac{{\rm d}#1}{{\rm d}r}}
\newcommand{\ddt}[1]{\frac{{\rm d}#1}{{\rm d}t}}
\newcommand{\deriv}[2]{\frac{\partial{#1}}{\partial{#2}}}
\newcommand{\derivtwo}[2]{\frac{\partial^2{#1}}{\partial{#2}^2}}
\newcommand{\derivate}[2]{\partial_{#2}{#1}}
\newcommand{\derivatetwo}[2]{\partial^2_{#2}{#1}}
% --> LM's additions :
\newcommand{\derivt}[1]{\dfrac{\partial{#1}}{\partial{t}}}
\newcommand{\derivphy}[2]{\dfrac{\partial{#1}}{\partial{\phyvar{#2}}}}
\newcommand{\derivpat}[2]{\dfrac{\partial{#1}}{\partial{\patvar{#2}}}}
\newcommand{\derivatet}[1]{\partial_t {#1}}
\newcommand{\derivatephy}[2]{\partial_{\phyvar{#2}}#1}
\newcommand{\derivatepat}[2]{\partial_{\patvar{#2}}#1}


%---> partial derivatives
\newcommand{\partialxi}{\,\partial_{x^i}}
\newcommand{\partialxj}{\,\partial_{x^j}}
\newcommand{\partialx}{\,\partial_{x^1}}
\newcommand{\partialy}{\,\partial_{x^2}}
\newcommand{\partialz}{\,\partial_{x^3}}

%---> gradient definitions
\newcommand{\gradxi}{\grad x^i}
\newcommand{\gradxGi}{\grad x_G^i}
\newcommand{\gradxj}{\grad x^j}
\newcommand{\gradxk}{\grad x^k}
\newcommand{\gradx}{\grad x^1}
\newcommand{\grady}{\grad x^2}
\newcommand{\gradz}{\grad x^3}
\newcommand{\gradr}{\grad r}
\newcommand{\gradtheta}{\grad \theta}
\newcommand{\gradphi}{\grad \varphi}
\newcommand{\gradB}{\grad B}
\newcommand{\gradF}{\grad F}
\newcommand{\gradG}{\grad G}

%-----> for the integrals
%\newcommand{\dd}{\:{\rm d}}
\newcommand{\dmu}{\:{\rm d}\mu}
\newcommand{\dphi}{\:{\rm d}\varphi}
\newcommand{\dr}{\:{\rm d}r}
\newcommand{\dt}{\:{\rm d}t}
\newcommand{\dtau}{\:{\rm d}\tau}
\newcommand{\dtaustar}{\:{\rm d}\tau^*}
\newcommand{\dtheta}{\:{\rm d}\theta}
\newcommand{\dvolume}{\:\jacobspace\dr\dtheta\dphi}
\newcommand{\dvelocity}{\:\jacobvel\dvGpar\dmu}
\newcommand{\dvspace}{\:{\rm d}^3\vecv}
\newcommand{\dvGpar}{\:{\rm d}\vGpar}
\newcommand{\dx}{\:{\rm d}x}
\newcommand{\dV}{\:{\rm d}V}

%---> surface average
%------> (theta,phi) average
\newcommand{\thetaphiavg}[1]{\langle\:{#1}\:\rangle_{\theta,\:\varphi}}
\newcommand{\TPavg}[1]{\langle\:{#1}\:\rangle_{\theta,\:\varphi}}
%------> flux surface average
\newcommand{\fluxSavg}[1]{\langle\:{#1}\:\rangle_{\rm Flux\:Surf}}
\newcommand{\FSavg}[1]{\langle\:{#1}\:\rangle_{\rm FS}}

%---> covariant and contravariant metric tensors and coordinates
\newcommand{\cotensor}{\{g_{ij}\}}
\newcommand{\contratensor}{\{g^{ij}\}}
\newcommand{\hatxGi}{\hat{x}_G^i}
\newcommand{\hatxi}{\hat{x}^i}
\newcommand{\hatgradxi}{\hat{\grad}x^i}
\newcommand{\muJgradxi}{\mu_0\vecJ\cdot\gradxi}
\newcommand{\muJgradxGi}{\mu_0\vecJ\cdot\gradxGi}
\newcommand{\vecbstari}{\mathbf{b}_s^{\ast i}}
\newcommand{\Bgradxi}{B\cdot\grad x^i}

%---> unit vectors
\newcommand{\er}{\vece_r}
\newcommand{\etheta}{\vece_\theta}
\newcommand{\ephi}{\vece_\varphi}
%\newcommand{\normvec}{\vec{n}}
\newcommand{\normvec}{\mathbf{n}}
\newcommand{\patnormvec}{\mathbf{\tilde{n}}}

%---> for quasi-neutrality solver
\newcommand{\rhos}{\rho_s}
\newcommand{\calL}{\mathcal{L}}
\newcommand{\phitilde}{\tilde{\phi}}
\newcommand{\FFTmPhi}{\hat{\Phi}^m(\hatr,\varphi)}
\newcommand{\FFTmvarrho}{\varrho^m(\hatr,\varphi)}
\newcommand{\sFFTmPhi}{\hatPhi^m}
\newcommand{\sFFTmvarrho}{\varrho^m}
\newcommand{\iFFTmPhi}[1]{\hatPhi^m_{#1}}
\newcommand{\iFFTmvarrho}[1]{\varrho^m_{\:#1}}
\newcommand{\FSavgphi}[1]{\FSavg{\hatphi}(\hatr_{#1})}
\newcommand{\Gammarhs}[1]{\Gamma(\hatr_{#1})}
\newcommand{\rmin}{r_{\rm min}}
\newcommand{\rmax}{r_{\rm max}}
\newcommand{\hatddr}[1]{\frac{{\rm d}#1}{{\rm d}\hatr}}
\newcommand{\Asum}{\hatA_{\rm sum}}

%---> for collision operator
\newcommand{\ninit}{n_{s_0}}
\newcommand{\ncoll}{\mathcal{N}_s}
\newcommand{\Bstar}{B^\ast_\parallel}
\newcommand{\Dpars}{\mathcal{D}_{\parallel s}}
\newcommand{\Dperps}{\mathcal{D}_{\perp s}}
\newcommand{\FbarMs}{\bar{F}_{M_s}}
\newcommand{\Tinit}{T_{s_0}}
\newcommand{\Tcoll}{T_{s, {\rm coll}}}
\newcommand{\Tcollpar}{T_{s, {\rm coll\:\parallel}}}
\newcommand{\Tcollperp}{T_{s, {\rm coll}\:\perp}}
\newcommand{\Tcollprime}{T_{s', {\rm coll}}}
\newcommand{\Vcoll}{V_{\parallel s, {\rm coll}}}
\newcommand{\Vpars}{\mathcal{V}_{\parallel s}}
%------> for normalised collision operator
\newcommand{\hatninit}{\hat{n}_{s_0}}
\newcommand{\hatnustars}{\hat{\nu}_{*s}}
\newcommand{\hatnustarsprime}{\hat{\nu}_{*s'}}
\newcommand{\hatDpars}{\hat{\mathcal{D}}_{\parallel s}}
\newcommand{\hatFbarMs}{\hat{\bar{F}}_{M_s}}
\newcommand{\hatTcoll}{\hat{T}_{s, {\rm coll}}}
\newcommand{\hatTcollprime}{\hat{T}_{s', {\rm coll}}}
\newcommand{\hatVcoll}{\hat{V}_{\parallel s, {\rm coll}}}
\newcommand{\hatVpars}{\hat{\mathcal{V}}_{\parallel s}}

%---> for collisional energy transfer operator
\newcommand{\Ttransf}{\langle T \rangle_{ss'}}
\newcommand{\Ttransfprime}{\langle T \rangle_{s's}}
\newcommand{\energys}{\mathcal{E}_s}
\newcommand{\energysprime}{\mathcal{E}_{s'}}
\newcommand{\Vtransf}{\langle V \rangle_{ss'}}
\newcommand{\Vtransfprime}{\langle V \rangle_{s's}}
\newcommand{\msprime}{m_{s'}}
\newcommand{\nsprime}{n_{s'}}
%------> for normalised energy transfer operator
\newcommand{\hatenergys}{\hat{\mathcal{E}}_s}
\newcommand{\hatenergysprime}{\hat{\mathcal{E}}_{s'}}
\newcommand{\hatnsprime}{\hat{n}_{s'}}
\newcommand{\hatvGpars}{\hatv_{G\|s}}
\newcommand{\hatvGparsprime}{\hatv_{G\|s'}}
\newcommand{\hatAsprime}{\hat{A}_{s'}}
\newcommand{\hatFbarsprime}{\hat{\bar{F}}_{s'}}
\newcommand{\hatTtransf}{\langle \hatT \rangle_{ss'}}
\newcommand{\hatTtransfprime}{\langle \hatT \rangle_{s's}}
\newcommand{\hatVtransf}{\langle \hatV \rangle_{ss'}}
\newcommand{\hatVtransfprime}{\langle \hatV \rangle_{s's}}

%---> for diffusion terms 
\newcommand{\Dcal}{\mathcal{D}}
\newcommand{\Dcalindx}[1]{\mathcal{D}_{#1}}

%---> for source terms
\newcommand{\barvGpars}{\bar{v}_{G_\parallel s}}
\newcommand{\barvGparsprime}{\bar{v}_{G_\parallel s^\prime}}
\newcommand{\barmu}{\bar{\mu}}
\newcommand{\JparB}{J_{\parallel B}}
\newcommand{\Tssrce}{T_{s, {\rm srce}}}
\newcommand{\Tssrceprime}{T_{s^\prime, {\rm srce}}}
\newcommand{\hatJpar}{\hatJ_{\parallel}}
\newcommand{\hatJparB}{\hatJ_{\parallel B}}
\newcommand{\hatTssrce}{\hat{T}_{s, {\rm srce}}}
\newcommand{\Sns}{S_{n_s}}
\newcommand{\Snsprime}{S_{n_{s^\prime}}}
\newcommand{\hatbarmu}{\hat{\bar{\mu}}}
\newcommand{\hatbarvGpars}{\hat{\bar{v}}_{G_\parallel s}}
\newcommand{\hatbarvGparsprime}{\hat{\bar{v}}_{G_\parallel s^\prime}}
\newcommand{\hatSns}{\hatS_{n_s}}
\newcommand{\hatSnsprime}{\hatS_{n_{s^\prime}}}
\newcommand{\hatTssrceprime}{\hatT_{s^\prime, {\rm srce}}}


%---> for time-splitting
\newcommand{\dxdt}{\frac{{\rm d}x}{{\rm d}t}}
\newcommand{\dydt}{\frac{{\rm d}y}{{\rm d}t}}
\newcommand{\dhdr}{\frac{{\rm d}h}{{\rm d}r}}

%---> for rbar definition
\newcommand{\rbar}{\bar{r}}

%-----> for the velocities
\newcommand{\vEer}{\vec{v}_{Es}\cdot\er}
\newcommand{\vEetheta}{\vec{v}_{Es}\cdot\etheta}
\newcommand{\vEgradr}{\vec{v}_{Es}\cdot\gradr}
\newcommand{\vEnulgradr}{\vec{v}_{Es_0}\cdot\gradr}
\newcommand{\vEgradtheta}{\vec{v}_{Es}\cdot\gradtheta}
\newcommand{\vEnulgradtheta}{\vec{v}_{Es_0}\cdot\gradtheta}
\newcommand{\vgcE}{\vec{v}_{Es_{GC}}}
\newcommand{\vgcEgradxi}{\vec{v}_{Es_{GC}}\cdot\gradxi}
\newcommand{\vgcEgradx}{\vec{v}_{Es_{GC}}\cdot\gradx}
\newcommand{\vgcEgrady}{\vec{v}_{Es_{GC}}\cdot\grady}
\newcommand{\vgcEgradz}{\vec{v}_{Es_{GC}}\cdot\gradz}
\newcommand{\vgcEgradr}{\vec{v}_{Es_{GC}}\cdot\gradr}
\newcommand{\vgcEgradtheta}{\vec{v}_{Es_{GC}}\cdot\gradtheta}
\newcommand{\vgcEgradphi}{\vec{v}_{Es_{GC}}\cdot\gradphi}
\newcommand{\vgcEetheta}{\vec{v}_{Es_{GC}}\cdot\etheta}
\newcommand{\vDgradxi}{\vec{v}_{Ds}\cdot\gradxi}
\newcommand{\vDgradr}{\vec{v}_{Ds}\cdot\gradr}
\newcommand{\vDgradtheta}{\vec{v}_{Ds}\cdot\gradtheta}
\newcommand{\vDgradphi}{\vec{v}_{Ds}\cdot\gradphi}
\newcommand{\vDer}{\vec{v}_{Ds}\cdot\er}
\newcommand{\vDetheta}{\vec{v}_{Ds}\cdot\etheta}
\newcommand{\vgcEgradB}{\vec{v}_{Es_{GC}}\cdot\gradB}

%---> for the energy conservation
\newcommand{\Ekin}{\delta\varepsilon_{\rm kin}}
\newcommand{\Epot}{\delta\varepsilon_{\rm pot}}
\newcommand{\Etot}{\delta\varepsilon_{\rm tot}}
\newcommand{\cte}{\operatorname{const}}

%---> for time splitting operator
\newcommand{\tildevGpar}{\tilde{v}_{G\parallel}}


\begin{document}

\title[]{ }

\author{}

\maketitle

\begin{abstract}



\end{abstract}


\vspace{0.1cm}

\noindent 
{\small\sc Keywords.}  {\small Finite difference method; Finite element method;  semi-Lagrangian scheme; Vlasov-Poisson model; Guiding-center model; Plasma physics.}


%\tableofcontents


\section{Introduction} 
\setcounter{equation}{0}
\label{sec:Intro}




%%%%%%%%%%%%%%%%%%%%%%%%%%%%%%%%%%%%%%%%%%%%%%%%%%%%%

\section{Theory}

%\label{sec:test}
\subsection{Spline}
\setcounter{equation}{0}

The accuracy of the Semi-Lagrangian method depends heavily on the interpolation method chosen. For example, for a cartesian grid is common to use cubic splines which have shown to give accurate results in an efficient manner. In our problem, with the hexagonal lattice, B-splines don't exploit the isotropy of the mesh (for more information see \cite{Mersereau79-IEEE}) and therefore we need a solution better adapted.  There are mainly two splines families that take advantage of the geometry's properties : hex-splines and the three directional box-splines. For a detailed comparison between these two types of splines we will refer to \cite{Condat2007}. Based on the latter, we have chosen to use box splines.

Let us describe such a model : we are given an initial sample $s[\veck] = \dist_0(\vecR \veck)$, where the points $\vecR \veck$ belong to our hexagonal mesh, and we need to know the values $f(X,V)$ where $(X,V) \notin \vecR \veck$. We want a spline surface $\dist(\vecx) = \sum c[\veck] \chi^n(\vecx - \vecR\veck)$ such that $f(\vecx)$ approximates $f_0(x,v)$ and where $\chi^n$ are the box-splines of compact hexagonal support and $c[\veck]$ are the box-splines coefficients which are obtained by \cite{Condat2006a}

\begin{equation}
\label{eqn:coefs}
 c = s * p
\end{equation}

where $*$ is the convolution operator, $s$ is the initial sample data and $p$ is a prefilter which will be defined later on.



\subsubsection{Three directional box-splines}

To construct the box-splines we will use the generator vectors $\mathbf{r_1, r_2, r_3}$ of the hexagonal lattice and we will introduce the box-splines basis functions $\varphi_{\Xi}(\vecx)$ where $\Xi = [\vecv_1 \vecv_2]$ which are defined as follows (\cite{Condat2006, Boor1993}):

\begin{equation}
\varphi_{\Xi} (\vecx) = \left\{
  \begin{array}{l l}
    \dfrac{1}{\lvert \det(\Xi)\rvert} & \quad \text{if $\Xi^{-1}\vecx \in [0,1)^2$ }\\
    0 & \quad \text{otherwise}
  \end{array} \right.
\end{equation}

and, for higher orders

\begin{equation} \label{eqn:boxsplines_basis}
\varphi_{\Xi \cup [v]}(\vecx) = \int_0^ 1 \varphi_\Xi(\vecx-t\vecv)dt
\end{equation}
  
  
We can define the Courant element \cite{Boor1993} as $\chi^1 = (\sqrt{3}/2)\varphi_{[\mathbf{r_1 r_2 -r_3}]}$ where $\mathbf{r_1, r_2, r_3}$ are the generator vectors. For higher orders we have the recursive expression : $\chi^n = (2/\sqrt{3})\chi^ {n-1} * \chi^ 1, \;\; n>1$ where the operator $*$ represents the convolution. For a complete analytical expression we refer to \cite{Condat2006} where we find the formula for $\chi^n(\vecx)$, which we have generalized to any hexagonal grid generated by a matrix $\vecR$ such that $\vecR = [\mathbf{r_1 r_2}] = \left[\begin{matrix}
r_{11}\\ 
r_{12}
\end{matrix}
\quad
\begin{matrix}
r_{21}\\  
r_{22}
\end{matrix}
\right]$. The generalized algorithm is as follows


\begin{align}
\label{eqn:boxspline_anal_formula}
\chi^n (x_1, x_2)  = &\sum_{k_1,k_2 = -n}^n \sum_{i =\max(k_1, k_2,0)}^{\min(k_1 + n, k_2 + n,n)} (-1)^ {k_1+k_2+i} \binom{n}{i-k_1}\binom{n}{i-k_2}\binom{n}{i} \nonumber \\  
&\sum_{d=0}^{n-1}  \binom{n-1+d}{d} \dfrac{1}{(2n-1+d)!(n-1-d)!} \nonumber \\
& \left| \dfrac{2}{\sqrt{3}} \left( x_2 - r_{12}k_1 -r_{22}k_2 \right) \right|^{n-1-d} \nonumber\\
& \left(x_1-r_{11}k_1 -r_{21}k_2 - \dfrac{1}{\sqrt{3}} \left| x_2 - r_{12}k_1 -r_{22}k_2 \right|\right)_+^{2n-1+d}
\end{align}

where $(x)^n_+ = \{ x^n \text{ for } x>0; \text{ } 0 \text{, otherwise}\}$.

This formula derives from a convolution between a particular Green function and a prefilter. For more information we refer to \cite{Condat2006}. In the latter we find as well an algorithm which exploits the twelve-fold symmetry of the mesh. Unfortunately this algorithm is specific to the second type of hexagon and doesn't take into account an eventual scaling. Nevertheless, if we denote by 
$\bar{\vecR} = \left[\begin{matrix}
\frac{1}{2}\\ 
-\frac{\sqrt{3}}{2}
\end{matrix}
\quad
\begin{matrix}
\frac{1}{2}\\ 
\frac{\sqrt{3}}{2}
\end{matrix}
\right]$
 the generating matrix of a second type hexagonal-mesh of spacing $1$, we can re-write the coordinates $\vecx$ in the basis $\vecR$ to the basis $\bar{\vecR}$ by using the formula : 

\begin{equation}
\label{eqn:change_basis}
\displaystyle \bar{\vecR}\vecR^{-1}\vecx = \bar{\vecx}
\end{equation} 

%
%\begin{verbatim}
%function val =
%\end{verbatim}




We will choose the definition in \eqref{eqn:boxspline_anal_formula} mostly for $n=2$ for higher orders we will opt for Box-MOMS (box-splines of maximum order and with minimal support) as presented in \cite{Condat2008}. The results for Box-MOMS of order 4 ($BM_4$) are encouraging specially when compared with normal Box-splines of the same order.





\subsubsection{Box splines coefficients}

How we determine the splines coefficients is almost as important as the splines themselves. We recall we have the formula \eqref{eqn:coefs}. Based on the literature available (notably \cite{Condat2007}) we have chosen for second-order box-splines the quasi-interpolation pre-filters $p_{IIR2}$ which seem to give the better results within a competitive time. The pre-filter $p_{IIR2}[i]$ of the point of local index $i$, for splines of order 2, is defined as follows : 


\begin{equation}
\label{eqn:filter}
p_{IIR2} [i] = \left\{
  \begin{array}{l l}
    1775/2304\text{,} & \quad \text{if } i = 0\\
    253/6912\text{,} & \quad \text{if } 0<i<7\\
    1/13824\text{,} & \quad \text{if } 6<i<19 \text{ and $i$ odd}\\
    11/6912\text{,} & \quad \text{if } 6<i<19 \text{ and $i$ even}\\
    0 & \quad \text{otherwise}
  \end{array} \right.
\end{equation}

Or for the splines of order 3 :

\begin{equation}
\label{eqn:filter}
p_{IIR2} [i] = \left\{
  \begin{array}{l l}
    244301/460800\text{,} & \quad \text{if } i = 0\\
    42269/576000\text{,} & \quad \text{if } 0<i<7\\
    -11809/6912000\text{,} & \quad \text{if } 6<i<19 \text{ and $i$ odd}\\
    1067/144000\text{,} & \quad \text{if } 6<i<19 \text{ and $i$ even}\\
    -23/576000\text{,} & \quad \text{if } 18<i<37 \text{ and ($k_1 = 0$ or $k_2 = 0$ or $k_1 = k_2$) }\\
    -109/288000\text{,} & \quad \text{if } 18<i<37 \\
    	-1/13824000\text{,} & \quad \text{if } 36<i<61 \text{ and ($k_1 = 0$ or $k_2 = 0$ or $k_1 = k_2$) }\\
    	97/6912000\text{,} & \quad \text{if } 36<i<61 \text{ and ($|k_1| = 2$ or $|k_2| = 2$) }\\
    	1/576000\text{,} & \quad \text{if } 36<i<61\\    		
    0 & \quad \text{otherwise}
  \end{array} \right.
\end{equation}

In details, let's give the exact formula for the coefficients. We suppose we have the functions $global(k_1, k_2) = i$ and $local(i,i_0) = j$ that give respectively the global index of $\vecx = \vecR \veck$ and the local index of that point regarding the point at position $i_0$.

\begin{equation}
c[\veck] = \sum_{\vecm \in \ZZd} s[\vecm] \cdot p_{IIR}[local(\vecm - \veck, \veck)]
\end{equation}

and using \eqref{eqn:filter} we obtain

\begin{align}
\label{eqn:coef_algo}
c[\veck] = \sum_{local(\vecm - \veck, \veck) = 0}^{18} s[\vecm] \cdot p_{IIR}[local(\vecm - \veck, \veck)]
\end{align}





\subsubsection{Optimizing the evaluation}

For the present state we have all the elements for the approximation of a function $\dist$ with second order box splines

\begin{equation}
\tilde{\dist}(\vecx) = \sum_{\veck \in \ZZd} c[\veck] \chi^2(\vecx - \vecR\veck)
\end{equation}

Even if we limit our sum to the vector $\veck$ that defines our domain, we would like to take advantage of the fact that the splines $\chi^2$ are only non-zeros in a limited number of points. Therefore we need to know the indices $\veck$ such that $\chi^2(\vecx - \vecR \vecx) \neq 0$. For this purpose we will use the strategy suggested in \cite{Condat2007} : to start we need to obtain the indices on the coordinate system generated by $\vecR$ : $\veck_0 = \left[ \lfloor u \rfloor \; \lfloor v \rfloor \right]$ where $\left[ u \; v \right]^T = \vecR^ {-1} \vecx $. Thus, in our case, with splines $\chi^2$ we only need 4 terms associated to the encapsulating rhomboid's vertices : $\vecR\veck_0$, $\vecR\veck_0 + \vecr_1$, $\vecR\veck_0 + \vecr_2$ and $\vecR\veck_0 + \vecr_1 + \vecr_2$. Finally we obtain :

\begin{align}
\label{eqn:chi2_fct_interpol}
\tilde{\dist}(\vecx) =& \;\;\;\;\; c[\veck_0] \; \chi^2(\vecx - \vecR\veck_0) \nonumber\\
	&+ c[\veck_0 +[1,0]]\; \chi^2(\vecx - \vecR\veck_0 - \vecr_1) \nonumber\\
	&+ c[\veck_0 +[0,1]]\; \chi^2(\vecx - \vecR\veck_0 - \vecr_2) \nonumber\\
	&+ c[\veck_0 +[1,1]]\; \chi^2(\vecx - \vecR\veck_0 - \vecr_1 - \vecr_2)
\end{align}

\rmk{1}  As the $\chi^2$ spline has a support of radius a unity, one of the elements of \eqref{eqn:chi2_fct_interpol} is null. But this formula allow us to keep a short general formula for all points on the mesh without having to compute the indices of the Voronoi cell to which $x$ belongs to.

%\begin{figure}[h!]
%  \begin{center}
%\begin{tabular}{ccc}
%
%\begin{tikzpicture}
%% Three directions of grid
%% vertical direction
%\draw (0,-2) -- (0,2) node[pos = 0.45, below left, text = blue]{};
%% upward
%\draw(-1.732,-1) -- (1.732,1);
%% downwards
%\draw(-1.732,1) -- (1.732,-1);
%% Hexagog
%\draw (0,-2) node[below, text = blue]{} -- (1.732,-1) node[right, text = blue]{} -- (1.732,1) node[right, text = blue]{} -- (0,2) node[above, text = blue]{}-- (-1.732,1) node[left, text = blue]{} -- (-1.732,-1) node[left, text = blue]{} -- (0,-2) ; % size 2
%%\draw (0,-1) -- (0.866,-.5) -- (0.866,.5) -- (0,1) -- (-.866,.5) -- (-.866,-.5) -- (0,-1) ; % size 1
%%\draw (0,-.5) -- (0.433,-.25) -- (0.433,.25) -- (0,.5) -- (-.433,.25) -- (-.433,-.25) -- (0,-.5) ; %size 0.5
%%\draw (0,-1.5) -- (1.3,-.75) -- (1.3,.75) -- (0,1.5) -- (-1.3,.75) -- (-1.3,-.75) -- (0,-1.5) ; % size 1.5
%\end{tikzpicture}
%\end{tabular}
%
%\end{center}
%  \caption{}
%  \label{fig:hex_cells}
%\end{figure}

\section{General algorithm}

To conclude we want to write the entire procedure.

\subsubsection*{Assumptions and initialization}

\begin{itemize}
	\item Mesh : defined by the matrix $\vecR$, its center $\vecx_0$ (typically the origin), its radius $L$ and the number of cells $N$;
	\item Points : The points of the mesh can be initialized as follows $ \vecx_i = \sum_{i} \vecR \veck_i$;
	\item Initial distribution : We assume we have a sample data such that $s^0[i] = \dist_0(\vecx_i)$ is given on the mesh points;
	\item Computing the characteristics : as the characteristic's feet are time-independent we can compute at the initialization step. We denote them $\tilde{\vecx}_i$.
\end{itemize}
	
\subsubsection*{Time loop}

\begin{itemize}
	\item Computing of the spline's coefficient : using the algorithm in \eqref{eqn:coef_algo} we compute the 19 elements sum on each point and pre-compute the spline coefficients using the pre-filter's values and the sample data $s^n$ ;
	\item Element by element interpolation :
		\begin{itemize}
			\item First we compute ${\veck}_0$ such that ${\veck}_0 = {\vecR}^ {-1} \tilde{\vecx}_i$. \rmk{2} We will need a test case to see if ${\veck_0} \in \Omega$ and that uses the boundary conditions to compute a new ${\veck_0}$ otherwise.
			\item Then we need to change the points' basis, such that they are defined in the spline basis. We use \eqref{eqn:change_basis} to find the coordinates of $\tilde{\vecx}_i$ in the basis of the second-type hexagonal mesh $\bar{\vecR}$, we will denote the solution $\bar{\vecx}_i$
			\item We use the formula \eqref{eqn:chi2_fct_interpol} to interpolate the value $\tilde{\dist}_0(\bar{\vecx}_i)$, the final formula is given below
			
\begin{align}
\label{eqn:chi2_fct_interpol2}
\tilde{\dist}_0(\bar{\vecx}_i) =& \;\;\;\;\; c[\veck_0] \; \chi^2(\bar{\vecR}\vecR^{-1}(\tilde{\vecx}_i - \vecR\veck_0)) \nonumber\\
	&+ c[\veck_0 +[1,0]]\; \chi^2(\bar{\vecR}\vecR^{-1}(\tilde{\vecx}_i - \vecR\veck_0 - \vecr_1)) \nonumber\\
	&+ c[\veck_0 +[0,1]]\; \chi^2(\bar{\vecR}\vecR^{-1}(\tilde{\vecx}_i - \vecR\veck_0 - \vecr_2)) \nonumber\\
	&+ c[\veck_0 +[1,1]]\; \chi^2(\bar{\vecR}\vecR^{-1}(\tilde{\vecx}_i - \vecR\veck_0 - \vecr_1 - \vecr_2))
\end{align}

		\end{itemize}
\end{itemize}




\subsection{Hermite finite element}
\setcounter{equation}{0}

Another possible way to interpolate is to use a 2d Hermite finite element \cite{zie}   

After the root $(X,V)$ of a characteristic is found and  the triangle in which it is located has been identified... 

There are ten degrees of freedom which are:
\begin{itemize}
\item[-] the values at the vertex of the triangle 
\item[-] the values of the derivatives 
\item[-] the values at the center of the triangle 
\end{itemize}


% mettre un dessin ici des degrés de liberté

\begin{equation}
\begin{cases}  

\partial _x f(x,y) = \partial _{H_1} f(x,y).   \partial _x H_1 +   \partial _{H_2} f(x,y).   \partial _x H_2  \\
\partial _y f(x,y) = \partial _{H_1} f(x,y).   \partial _y H_1 +   \partial _{H_2} f(x,y).   \partial _y H_2             

\end{cases}
\end{equation}

With $H_1$ et $H_2$ the  hexaedric coordinates. Since   

\begin{equation}
\begin{cases}  

\displaystyle{x = \frac{ H_1 - H_2 }{\sqrt{3}} },  \\[2mm]
\displaystyle{y =  H_1 + H_2 },        

\end{cases}
\end{equation}

We obtain :
\begin{equation}
	\begin{cases}  
       \displaystyle{  \partial _x H_1 = \frac{\sqrt{3}}{ 2} ~;~  \partial _y H_1 = \frac{1}{ 2}  }\\[2mm]
       \displaystyle{  \partial _x H_2 = \frac{-\sqrt{3}}{2} ~;~  \partial _y H_2 = \frac{1}{ 2}  } . 
\end{cases}	
\end{equation}

At the vertexes the values are given and $f(G) = \frac{f(S_1)+f(S_2)+f(S_3)}{3}$. As for the values of the derivatives, we use the finite difference method along the hexagonal directions as it can be seen in figure (?)

%%%%%%%%%%%%%%&%%%%%%%%%%%%%%%%%%%%%%%%%%%%%%%%%%%%%%%

\section{Numerical tests}
%\label{sec:test}
\setcounter{equation}{0}


%\begin{figure}[!]
%
% \includegraphics[width=3.5cm]{a02_t5.png}
%	
%  \caption{\label{fig:res}}
%
%\end{figure}


\section{Conclusion and perspectives}
\label{sec:conc}
\setcounter{equation}{0}

In this paper 

\bibliographystyle{plain}
\begin{thebibliography}{99} 

\bibitem{son}
{\sc E. Sonnendrücker, J. Roche, P. Bertrand, A. Ghizzo } The Semi-Lagrangian Method for the Numerical Resolution of Vlasov Equations, {\it Comput. Phys },{\bf 149, 201–220} (1999)  
  
  
\bibitem{zie}
{\sc Gustavo Buscaglia, Vitoriano Ruas } Finite element methods for the Stokes system based on a
Zienkiewicz type N-simplex, {\it Comput. Methods Appl. Mech. Engrg.},{\bf 272, 83-99} (2014)  
  

\end{thebibliography}



 

\end{document}

