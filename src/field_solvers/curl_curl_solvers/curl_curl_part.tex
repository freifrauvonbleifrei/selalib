\documentclass[12pt,a4paper,twoside]{article}
\usepackage{graphicx}
%\usepackage{epstopdf}
%\usepackage{subcaption}
%\usepackage{algorithmic}

%\usepackage{geometry}	


\usepackage{authblk}

%\usepackage{ifthen}
%\usepackage[l2tabu,orthodox]{nag}
%\usepackage[numbers,square]{natbib}
%\usepackage{abstract}
%\usepackage{showlabels}

\renewcommand\Affilfont{\fontsize{9}{10.8}\itshape}

\DeclareRobustCommand{\katharina}[1]{ {\begingroup\sethlcolor{cyan}\hl{[Katharina: #1]}\endgroup} }

\setlength{\voffset}{-28.4mm}
\setlength{\hoffset}{-1in}
\setlength{\topmargin}{20mm}
\setlength{\oddsidemargin}{25mm}
\setlength{\evensidemargin}{25mm}
\setlength{\textwidth}{160mm}

\setlength{\parindent}{0pt}

\setlength{\textheight}{235mm}
\setlength{\footskip}{20mm}
\setlength{\headsep}{50pt}
\setlength{\headheight}{0pt}
\usepackage{amssymb,amsmath,amsfonts,amstext,amsthm}
\usepackage{bbm}
\usepackage{latexsym}
\usepackage{multirow}
\usepackage[english]{babel}
\usepackage{epstopdf}
\usepackage{subcaption}

\usepackage[bbgreekl]{mathbbol}

%
\usepackage{tikz}
\usepackage{tikz-cd}
\usetikzlibrary{arrows,matrix,shapes,positioning}
\usetikzlibrary{calc,patterns,shapes.geometric}
\usetikzlibrary{decorations.markings,decorations.pathmorphing,decorations.pathreplacing}
\usetikzlibrary{decorations.text,decorations.shapes}



\usepackage{graphicx}
\usepackage{xcolor}
\usepackage{soul}
\usepackage[colorlinks=true,breaklinks=true]{hyperref}
\usepackage{url}




\let\divby\div
\let\div\undefined
\DeclareMathOperator{\div}{div}
\DeclareMathOperator{\curl}{curl}
\DeclareMathOperator{\grad}{grad}

\DeclareMathOperator{\Id}{Id}
\DeclareMathOperator{\diag}{diag}
\let\L\undefined
\newcommand{\L}{\mathrm{L}}
\newcommand{\K}{\mathrm{K}}
\newcommand{\du}{\,\mathrm{d}}
\newcommand{\R}{\mathbb{R}}
\newcommand{\N}{\mathbb{N}}
\newcommand{\ab}{{\mathbf{a}}}
\newcommand{\yb}{{\mathbf{y}}}
\newcommand{\zb}{{\mathbf{z}}}
\newcommand{\ub}{{\mathbf{u}}}
\newcommand{\xb}{{\mathbf{x}}}
\newcommand{\vb}{{\mathbf{v}}}
\newcommand{\Xb}{{\mathbf{X}}}
\newcommand{\Vb}{{\mathbf{V}}}
\newcommand{\eb}{{\mathbf{e}}}
\newcommand{\jb}{{\mathbf{j}}}
\newcommand{\pb}{{\mathbf{p}}}
\newcommand{\qb}{{\mathbf{q}}}
\newcommand{\rb}{{\mathbf{r}}}
\newcommand{\Ab}{{\mathbf{A}}}
\newcommand{\Bb}{{\mathbf{B}}}
\newcommand{\Db}{{\mathbf{D}}}
\newcommand{\Eb}{{\mathbf{E}}}
\newcommand{\Fb}{{\mathbf{F}}}
\newcommand{\Gb}{{\mathbf{G}}}
\newcommand{\Hb}{{\mathbf{H}}}
\newcommand{\Rb}{{\mathbf{R}}}
\newcommand{\Sb}{{\mathcal{S}}}
\newcommand{\bb}{{\mathbf{b}}}
\newcommand{\bt}{\mathbf{t}}
\newcommand{\bn}{\mathbf{n}}
\newcommand{\rhob}{{\boldsymbol{\rho}}}
\newcommand{\phib}{{\boldsymbol{\phi}}}
\newcommand{\xib}{{\boldsymbol{\xi}}}
\newcommand{\Xib}{{\boldsymbol{\Xi}}}
\newcommand{\tv}{{\tilde{v}}}
\newcommand{\tV}{{\tilde{V}}}
\newcommand{\tVb}{{\tilde{\mathbf{V}}}}
\newcommand{\tvb}{{\tilde{\mathbf{v}}}}
\newcommand{\hVb}{{\hat{\mathbf{V}}}}
\newcommand{\hvb}{{\hat{\mathbf{v}}}}
\newcommand{\hv}{{\hat{v}}}
\newcommand{\tub}{\tilde{\mathbf{u}}}
\newcommand{\teb}{{\tilde{\mathbf{e}}}}
\newcommand{\te}{{\tilde{e}}}
\newcommand{\tE}{{\tilde{E}}}
\newcommand{\ta}{{\tilde{a}}}
\newcommand{\tA}{{\tilde{A}}}
\newcommand{\tjb}{{\tilde{\mathbf{j}}}}
\newcommand{\tEb}{{\tilde{\mathbf{E}}}}
\newcommand{\tBb}{{\tilde{\mathbf{B}}}}
\newcommand{\tAb}{{\tilde{\mathbf{A}}}}
\newcommand{\tPhi}{{\tilde{\Phi}}}
\newcommand{\tphi}{{\tilde{\phi}}}
\newcommand{\tab}{\tilde{\mathbf{a}}}
\newcommand{\tbb}{{\tilde{\mathbf{b}}}}
\newcommand{\tb}{{\tilde{b}}}
\newcommand{\tB}{{\tilde{B}}}
\newcommand{\tj}{{\tilde{j}}}
\newcommand{\tdb}{\tilde{\mathbf{d}}}
\newcommand{\tkb}{\tilde{\mathbf{k}}}
\newcommand{\tlb}{\tilde{\mathbf{\ell}}}
\newcommand{\trho}{{\tilde{\rho}}}
\newcommand{\trhob}{{\tilde{\boldsymbol{\rho}}}}
\newcommand{\tphib}{{\tilde{\boldsymbol{\phi}}}}
\newcommand{\tf}{\tilde{f}}
\newcommand{\tJb}{\tilde{\mathbf{J}}}
\newcommand{\Jb}{\mathbf{J}}
\newcommand{\tvpb}{\boldsymbol{\tilde{\varphi}}}
\newcommand{\tvp}{\tilde{\varphi}}
\newcommand{\BB}{{\ensuremath{\mathbb{B}}}}
\newcommand{\MM}{\ensuremath{\mathbb{W}}}
\newcommand{\WM}{\ensuremath{\mathbb{W}}}
\newcommand{\M}{\ensuremath{\mathbb{M}}}
\newcommand{\GM}{\ensuremath{\mathbb{G}}}
\newcommand{\LM}{\ensuremath{\mathbb{L}}}
\newcommand{\JJ}{\ensuremath{\mathbb{J}}}
\newcommand{\tM}{\ensuremath{\mathsf{\tilde{M}}}}
\newcommand{\C}{\ensuremath{\mathsf{C}}}
\newcommand{\D}{\ensuremath{\mathsf{D}}}
\newcommand{\G}{\ensuremath{\mathsf{G}}}
\newcommand{\tBB}{\ensuremath\tilde{{\mathbb{B}}}}
\newcommand{\tJJ}{\ensuremath\tilde{{\mathbb{J}}}}
\newcommand{\NN}{\ensuremath{\mathbb{N}}}
\newcommand{\DF}{\ensuremath{\mathbb{DF}}}
\newcommand{\tom}{\ensuremath{\tilde{\Omega}}}
\newcommand{\tLa}{\ensuremath{\tilde{\Lambda}}}
\newcommand{\tLam}{\ensuremath{\tilde{\Lambda}}}
\newcommand{\tLab}{\ensuremath{\tilde{\boldsymbol{\Lambda}}}}
\newcommand{\tLaB}{\ensuremath{\tilde{\mathbf{\Lambda}}}}
\newcommand{\tLaBB}{\ensuremath{\tilde{\mathbb{\Lambda}}}}
\newcommand{\Lab}{\ensuremath{\boldsymbol{\Lambda}}}
\newcommand{\LaB}{\ensuremath{\mathbf{\Lambda}}}
\newcommand{\LaBB}{\ensuremath{\mathbb{\Lambda}}}
%
\newcommand{\Ord}{\mathcal{O}}

\newcommand{\im}{\mathrm{i}}
\newcommand{\e}{\mathrm{e}}
\newcommand{\mb}{{\mathbf{m}}}
\newcommand{\db}{\mathbf{d}}
\newcommand{\kb}{\mathbf{k}}

\newcommand{\tMM}{\ensuremath\tilde{{\mathbb{W}}}}
\newcommand{\tC}{\ensuremath\tilde{{\mathbb{C}}}}
\newcommand{\tD}{\ensuremath\tilde{{\mathbb{D}}}}
\newcommand{\tG}{\ensuremath\tilde{{\mathbb{G}}}}



\newcommand{\nr}{n_{\mathrm{ref}}}
\newcommand{\qr}{q_{\mathrm{ref}}}
\newcommand{\Tr}{T_{\mathrm{ref}}}
\newcommand{\mr}{m_{\mathrm{ref}}}
\newcommand{\Br}{B_{\mathrm{ref}}}
\newcommand{\betar}{\beta_{\mathrm{ref}}}
\newcommand{\Lr}{L_{\mathrm{ref}}}
\newcommand{\vr}{v_{\mathrm{ref}}}
\newcommand{\omr}{\omega_{\mathrm{ref}}}
\newcommand{\rhor}{\rho_{\mathrm{ref}}}
\newcommand{\tht}{\hat{t}}
\newcommand{\xht}{\hat{\xb}}
\newcommand{\vht}{\hat{\vb}}
\newcommand{\muht}{\hat{\mu}}
\newcommand{\vpht}{\hat{\vp}}
\newcommand{\epsr}{\varepsilon_{\mathrm{ref}}}
\newcommand{\phiht}{\hat{\phi}}
\newcommand{\psigaht}{\hat{\psiga}}
\newcommand{\Apht}{\hat A_{\parallel}}
\newcommand{\Tsht}{\hat{T_{s}}}
\newcommand{\fht}{\hat{f}}
\newcommand{\qht}{\hat{q}}
\newcommand{\mht}{\hat{m}}
\newcommand{\Eht}{\hat{\Eb}}
\newcommand{\Bht}{\hat{\Bb}}
\newcommand{\rhoht}{\hat{\rho}}
\newcommand{\jht}{\hat{\jb}}
\newcommand{\fsht}{\hat{f_s}}

\newcommand{\Edh}{\widehat{\delta \Eb}}
\newcommand{\Bdh}{\widehat{\delta \Bb}}
\newcommand{\fdh}{\widehat{\delta f}}
\newcommand{\Jdh}{\widehat{\delta \Jb}}
\newcommand{\rhodh}{\widehat{\delta \rho}}
\newcommand{\phidh}{\widehat{\delta \Phi}}

\newtheorem{theorem}{Theorem}[section]
\newtheorem{prop}[theorem]{Proposition}
\newtheorem{mydef}[theorem]{Definition}
\newtheorem{lemma}[theorem]{Lemma}
\newtheorem{cor}[theorem]{Corrolary}
\newtheorem{remark}[theorem]{Remark}


\numberwithin{equation}{section}

\begin{document}
\section{Curl-Curl equation}
\begin{align*}
\nabla \times \nabla \times \Ab(\xb,t) + \nabla p(\xb,t) &= \Jb(\xb,t), \\
\nabla \cdot \Ab &= 0.
\end{align*}


weak formulation for test functions $\phib \in H(\curl,\Omega),\psi \in H^1(\Omega)$:
\begin{align*}
\int \nabla \times \phib \cdot \nabla \times \Ab \du \xib + \int \phib \cdot \nabla p \du \xb &= \int \phib \cdot \Jb \du \xb, \\
-\int \nabla \psi \cdot \Ab \du \xb &= 0,
\end{align*}
where $\Ab \in H(\curl,\Omega), p \in H^1(\Omega)$.

Use finite element ansatz: $\Ab = \Lab^1(\xb) \ab, p = \Lambda^0(\xb) \pb$ satisfying the de Rham sequence
\begin{align*}
\nabla \Lambda^0 = \Lab^1 \G, \nabla \times \Lab^1 = \Lab^2 \C
\end{align*}
and $ \C \G \psi = 0 \forall \psi \in H^1(\Omega), \D \C \phib \forall \phib \in  H(\curl,\Omega), \C^\top \D^\top \psi \forall \psi  \in {L^2}^\star(\Omega), \G^T \C^T \phib \forall \phib \in  H^\star(\div,\Omega)$ for the derivative matrices $\G, \C, \D$ 
 to obtain
\begin{align*}
\int \Lab^2 \C \cdot \Lab^2 \C \ab \du \xb + \int \Lab^1 \cdot \Lab^1 \G \pb \du \xb &= \int \Lab^1 \cdot \Jb \du \xb, \\
\int \Lab^1 \G \cdot \Lab^1 \ab \du \xb &= 0.
\end{align*}

Introducing the mass matrices $M_1 = \int  \Lab^1 \cdot \Lab^1 \du \xb, M_2 = \int  \Lab^2 \cdot \Lab^2 \du \xb$ we get
\begin{align*}
\C^\top M_2 \C \ab + M_1 \G \pb &= \jb , \\
\G^\top M_1 \ab &= 0.
\end{align*} 

For this system we use the Uzawa iteration with the operators $\K =  \C^\top M_2 \C, \L= M_1 \G , \L^\top = \G^\top M_1$:
\begin{align*}
\K \ab + \L \pb &= \jb, \\
\L^\top \ab &= 0.
\end{align*}

Because of the de Rham complex the right hand side of a multiplication of the $A$ operator will alway be weakly divergence free $\G^T A\ab = 0$. However, the $L^2$ projection $M_1^{-1} \int \Lab^1 \cdot J(\xb) \du \xb$ of analytically divergence free functions may not be numerically divergence free. Therefore, we use a constraint in the solver of the $A$ operator, which is symmetric so that we can still use a CG solver
\begin{align*}
\K_r= \C^\top M_2 \C + r M_1 \G \G^\top M_1 
\end{align*}
\subsection{Uzawa algorithm}
Initialisation: choose start value $\pb^0$
\begin{align*}
\ab^0 &= \K_r^{-1}(\jb-\L \pb^0), \\
\rb_0^0 &= \L^\top \ab^0, \\
\qb_0^0 &= \rb_0^0. 
\end{align*}

Iteration $n=0,..., N-1$ :
\begin{align*}
\qb_1 &= \K_r^{-1} \L \qb_0^n,\\
\db_0 &=\L^\top \qb_1, \\
\alpha &= \frac{\qb_0^n \cdot \db_0 }{\qb_0^n \cdot \rb_0^n}, \\
\pb^{n+1} &= \pb^n + \alpha \qb_0^n, \\
\rb_0^{n+1} &= \rb_0^n - \alpha \db_0, \\
\ab^{n+1} &= \ab^n - \alpha \qb_1, \\
\beta &= \frac{\rb_0^{n+1} \cdot \db_0}{\qb_0^n \cdot \db_0}, \\
\qb_0^{n+1} &= \rb_0^{n+1} - \beta \qb_0^n.
\end{align*}
If $\|\rb_0^{n+1}\| < $ tol or $\|\qb_0^n\| < $ tol  the iterations ends.

\subsection{FFT solver}
Since the iterative solution of the inverse of $\K_r$ in the Uzawa algorithm is quite expensive, we search for a direct solver. We note that the discrete mass matrices have block-diagonal form,
\begin{align*}
M_1 = \begin{pmatrix}
M_{11}& 0 &0 \\ 0&M_{12}&0 \\ 0&0&M_{13}
\end{pmatrix}, \M_2 = \begin{pmatrix}
M_{21}& 0 &0 \\ 0&M_{22}&0 \\ 0&0&M_{23}
\end{pmatrix}
\end{align*}
and the curl and gradient matrices have the following block structure
\begin{align*}
\C= \begin{pmatrix}
0& -D_3 & D_2 \\ D_3&0&-D_1 \\ -D_2&D_1&0
\end{pmatrix}, \G= \begin{pmatrix}
D_1 \\ D_2 \\ D_3
\end{pmatrix},
\end{align*}
where $D_i, i=1,2,3$ denotes the derivative matrix along direction i. So we can compute
\begin{align*}
\C^T M_2 \C &= \begin{pmatrix}
D_3^\top M_{22} D_3 + D_2^\top M_{23} D_2 & -D_2^\top M_{23}D_1& -D_3^\top M_{22}D_1
\\
-D_1^\top M_{23}D_2 &D_3^\top M_{21} D_3 + D_1^\top M_{23} D_1 & -D_3^\top M_{21}D_2 
 \\
-D_1^\top M_{22}D_3 & -D_2^\top M_{21}D_3&D_2^\top M_{21} D_2 + D_1^\top M_{22} D_1
\end{pmatrix},\\
M_1 \G \G^\top M_1 &= \begin{pmatrix}
M_{11}D_1 D_1^\top M_{11} & M_{11}D_1 D_2^\top M_{12} &M_{11}D_1 D_3^\top M_{13} 
\\
M_{12}D_2 D_1^\top M_{11} & M_{12}D_2 D_2^\top M_{12} & M_{12}D_2 D_3^\top M_{13} 
\\
M_{13}D_3 D_1^\top M_{11} & M_{13}D_3 D_2^\top M_{12} & M_{13}D_3 D_3^\top M_{13}
\end{pmatrix}. 
\end{align*}

These operators can be inverted directly in Fourier space. The eigenvalues of the derivative matrices $D_i,i=1,2,3$ and the circulant mass matrices $M_{ij}, i=1,2,j=1,2,3$ are given by
\begin{itemize}
\item $D: \lambda_k^+=\frac{1}{\Delta x}\left(1-\exp\left(-\frac{2\pi i k}{N} \right) \right), k = 0, ...,N-1$. 
\item $D^\top: \lambda_k^-=\frac{1}{\Delta x}\left(1-\exp\left(\frac{2\pi i k}{N} \right) \right), k = 0, ...,N-1$.
\item $M$ with row $( c_p,...,c_0,...,c_p)$ and splines of order $p$: $\lambda_k^{(p)}= c_0 + \sum_{j=1}^p 2 c_j \cos\left(\frac{2 \pi k j}{N}\right) $.
\end{itemize}
After Fourier transformation, we have a $3 \times 3$ system for each Fourier mode, which can be solved explicitly. When we have a non-periodic system with clamped boundary conditions and a metric in the mass matrices, we can still use the Fourier solver as a preconditioner for a CG solver.

\subsection{Jacobi solver}
Another option for a direct solver would be a Jacobi solver, where we use just the main diagonal of the curl matrix $\K_r$. We have to check how exact the inversion of the $\K_r$ matrix has to be in order to lead to a convergence of the Uzawa algorithm.

\subsection{Self-manufactured solution}

First option:
\begin{align*}
\Ab(\xb)&=\begin{pmatrix} \cos(x_1+x_2+x_3)\\
0 \\-\cos(x_1+x_2+x_3)
\end{pmatrix},\\
\nabla \times \nabla \times \Ab(\xb)&=\begin{pmatrix} 3 \cos(x_1+x_2+x_3)\\
0\\ -3 \cos(x_1+x_2+x_3)
\end{pmatrix}, \\
p(x) &= 0.
\end{align*}

Second option:
\begin{align*}
\Ab(\xb)&=\begin{pmatrix} \sin(x_1+x_2+x_3)+\cos(x_1+x_2+x_3)\\
-\sin^2(x_1+x_2+x_3)+\cos^2(x_1+x_2+x_3) -\cos(x_1+x_2+x_3)\\
(\sin(x_1+x_2+x_3)-1)\sin(x_1+x_2+x_3)-\cos^2(x_1+x_2+x_3) 
\end{pmatrix},\\
\nabla \times \nabla \times \Ab(\xb)&=\begin{pmatrix} 3( \sin(x_1+x_2+x_3)+ \cos(x_1+x_2+x_3) )\\
 -3( \cos(x_1+x_2+x_3)-4 \cos(2(x_1+x_2+x_3)) )\\ 
 -3( \sin(x_1+x_2+x_3)+4 \cos(2(x_1+x_2+x_3)) )
\end{pmatrix}, \\
p(\xb) &= \cos(x_1+x_2+x_3), \\
\nabla p(\xb) &= \begin{pmatrix}
-\sin(x_1+x_2+x_3)\\-\sin(x_1+x_2+x_3)\\-\sin(x_1+x_2+x_3)
\end{pmatrix}.
\end{align*}

Then our right-hand side is computed as
\begin{align*}
\Jb(\xb) = \nabla \times \nabla \times \Ab(\xb) + \nabla p(\xb).
\end{align*}

Since the right-hand side is usually accumulated from the particle distribution function, we introduce artifical particle noise by making a convex combination of the analytical current and a three dimensional random noise vector:
\begin{align*}
\tilde \Jb(\xb) = (1-\epsilon)\Jb(\xb) + \epsilon \operatorname{random}[0,1].
\end{align*}

%
%
%\subsection{alternative algorithm}
%Initialisation: choose start value $\pb^0$
%\begin{align*}
%\ab^0 &= A^{-1}(\jb-B\pb^0), \\
%\qb_0^0 &= B^\top \ab^0. 
%\end{align*}
%
%Iteration $n=0,..., N-1$ :
%\begin{align*}
%\db_0 &= B^\top \ab^n, \\
%\qb_1 &= A^{-1} B \qb_0^n,\\
%\yb_0 &=B^\top \qb_1, \\
%\alpha &= \frac{\db_0 \cdot \db_0 }{\db_0 \cdot \yb_0}, \\
%\pb^{n+1} &= \pb^n + \alpha \qb_0^n, \\
%\ab^{n+1} &= \ab^n + \alpha \qb_1, \\
%\rb_0 &=B^\top \ab^{n+1}, \\ 
%\beta &= \frac{\rb_0 \cdot \rb_0}{\db_0 \cdot \db_0}, \\
%\qb_0^{n+1} &= \rb_0^{n+1} + \beta \qb_0^n.
%\end{align*}
%If $\|\rb_0\| < $ tol the iterations ends.

\end{document}